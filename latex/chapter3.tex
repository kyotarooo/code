\chapter{実験装置}
本研究では,グリースの温度依存性を評価するために,密着強度およびちょう度の測定を各温度条件下で実施した.具体的には,-20\,${}^\circ$Cから180\,${}^\circ$Cの範囲において,以下の2種類の実験を行った.まず,JIS K 2220に規定された1/2円錐を用いた不混和ちょう度試験により,温度によるちょう度の変化を測定した.次に,グリースとナイロン66との密着強度の温度依存性を評価する実験を実施した.以下に,各試験方法について詳細に述べる.

\section{実験に使用したグリース}
本研究では,4種類のグリースを試料として使用した.各グリースは,基油の種類や増ちょう剤が異なる代表的な市販製品であり,それぞれGrease A 〜Grease Dとして記載する.表\ref{teble:グリースの物性}に,各グリースの一般的な物性を示す.なお,一部の項目はメーカー公表値を引用した.表に示した代表物性は参考値であり,本研究で使用した実験条件下での特性は後述の実験結果に示す.

\begin{table}[htbp]
\begin{center}
\caption{General properties of greases used in this study.}\label{teble:グリースの物性} 
\begin{threeparttable}
\begin{tabular}{c|c|c|c|c}
\hline\hline
Grease & Base\ oil & Thickener & Worked\ pen.\tnote{(a)} \ [dmm] & Temp.\ range \ [\si{\celsius}] \\
\hline
A & Mineral\ oil  & Li\ soap & 250\ (No.3) &-50〜150\\
B & Synthetic\ hydrocarbon\ oil & Li\ soap & 220〜250\ (No.3) &-50〜150\\
C & Mineral\ oil & Urea & 325\ (No.1) & -20〜200\\
D & Synthetic\ hydrocarbon\ oil & Li+Na\ soap & 272\ (No.2) &-49〜150\\
\hline
\end{tabular}
\begin{tablenotes}
\footnotesize
\item[(a)] Worked penetration (pen.) measured with a standard cone at 25~\si{\celsius} after 60 strokes (JIS K 2220)\cite{ちょう度JIS}.
\item Representative physical properties of each grease are based on manufacturers' technical data sheets (1)–(4).
\end{tablenotes}
\end{threeparttable}
\end{center}
\end{table}\par

%%グリースA=SRL,グリースB=LRL,グリースC=エクセライト,グリースD=マルティノック

\section{1/2円錐を用いたちょう度測定試験}
本試験では,グリースのちょう度を評価するため,JIS K 2220に準拠した1/2円錐を用いた不混和ちょう度試験を実施した.規格では25\,${}^\circ$Cにおける測定方法が定められているが,本研究では温度によるちょう度の変化を確認するため,25\,${}^\circ$Cを基準として,それ以上の複数の温度条件下でも同一の手法により測定を行った.
\subsection{JIS K 2220に準拠した1/2円錐を用いた不混和ちょう度試験の原理}
本試験は,規定温度(25\,${}^\circ$C)において,JIS K 2220に準拠した1/2円錐を用いて,グリースの不混和ちょう度を測定するものである.円錐を試料が充填された金属容器内に自由落下させ,5秒間静置後の進入深さを0.1\,mm単位で測定し,その値を10倍したものをちょう度とする\cite{ちょう度JIS}.測定値は,JIS Z 8401に定められた丸め規則Bに従い,丸め幅1で表記した\cite{ちょう度JIS,丸め方}.\par
なお,1/2円錐を用いたちょう度試験は,標準円錐を用いたちょう度試験の代替を目的としたものではなく,異なる条件下での測定手法として位置付けられている点に留意する必要がある\cite{ちょう度JIS}.また,不混和ちょう度は,使用中のグリースの硬さを混和ちょう度ほど明確に反映しないことが知られている\cite{ちょう度JIS}.

\subsection{1/2円錐を用いたちょう度測定の試験装置および器具}
図\ref{fig:ちょう度測定}に,本実験で使用した1/2円錐を用いたちょう度測定試験装置の外観を示す.装置は,円錐保持機構,静圧軸受,デジタルカメラ,および自作の支持構造から構成される.
円錐およびその保持軸には,離合社製(型式872)の1/2円錐試験器を用いた.保持軸の滑らかな自由落下を実現するため,静圧軸受には東海カーボン製のグラファイトを採用した.円錐の進入深さの記録には,Sony製デジタルカメラZV-1Fで図\ref{fig:ちょう度測定}に示す通り,真横から持ち上げた1/2円錐を撮影し,画像解析ソフトImageJにより解析を行った.

\begin{figure}[htbp]
  \centering
  \includegraphics[width=105mm]{images/ちょう度測定外観図.pdf} %pdfのみ
  \caption{External view of the penetration measurement test device.}\label{fig:ちょう度測定} 
\end{figure}\par

\newpage
\vspace{5mm}
\noindent\textbf{1/2円錐}\par
本試験で使用した1/2円錐は,離合社製(型式872)のものであり,JIS K 2220に基づいて設計・校正されており,質量は22.50\,gである.図\ref{fig:円錐の寸法}に1/2円錐の寸法を示す.\par

\begin{figure}[htbp]
  \centering
  \includegraphics[width=36mm]{images/円錐寸法.pdf} %pdfのみ
  \caption{Dimensions of the 1/2 cone used in the experiment. All dimensions in the diagram are in mm.}\label{fig:円錐の寸法} 
\end{figure}\par

\vspace{5mm}
\noindent\textbf{試料を入れる金属容器}\par
試料(グリース)を充填する金属容器は,円筒と底板を瞬間接着剤で隙間なく接合して作製した.図\ref{fig:金属容器}に示すように,内径38.1\,mm,高さ31.8\,mmの円筒と底板を瞬間接着剤で接合して作製した.底板は,隙間からの漏れを防ぐために丁寧に接着した.測定後はハンマーで容器を分解し,各部品を洗浄・乾燥のうえ,再接着して再使用した.\par

\begin{figure}[htbp]
  \begin{minipage}[b]{0.31\linewidth}
    \centering
    \includegraphics[width=\linewidth,keepaspectratio]{images/金属容器全体図.pdf}
    \subcaption{Overall view of metal container.}
  \end{minipage}
  \begin{minipage}[b]{0.31\linewidth}
    \centering
    \includegraphics[width=\linewidth,keepaspectratio]{images/金属円筒寸法.pdf}
    \subcaption{Dimensions of hollow metal cylinder.}
  \end{minipage}
   \begin{minipage}[b]{0.34\linewidth}
    \centering
    \includegraphics[width=\linewidth,keepaspectratio]{images/底板.pdf}
    \subcaption{Bottom plate dimensions.}
  \end{minipage}
  \caption{Overview of metal containers based on JIS K 2220. All dimensions in the diagram are in mm.}\label{fig:金属容器} 
\end{figure}\par

\newpage
\vspace{5mm}
\noindent\textbf{恒温槽}\par
試料の加熱および低温保持には,恒温槽を用いた.高温域(20〜180\si{\celsius})の試験には恒温乾燥器 ONW-300(アズワン株式会社製)を2台使用し,異なる温度条件下で同時に試験を実施した.本装置の槽内温度範囲は 20~300 \si{\celsius} であり,温度制御にはマイコン式 PID 制御が用いられている.設定温度に対する精度は,300 \si{\celsius} 設定時において庫内無負荷時の温度分布幅が 12 \si{\celsius} である.測定前には,設定温度に達してから 40 分間以上温度を安定させた後に試料を投入した.試料は槽内中央に設置し,扉の開閉による温度変動を避けるため,観察および操作は最小限に留めた.\par
一方,低温域(-20~20 \si{\celsius})の試験には,プログラム温湿度調整器 HP-103(いすゞ製作所製)を使用した.この装置は湿度制御機能も備えているが,本研究では温度制御のみを利用した.

\vspace{5mm}
\noindent\textbf{K型熱電対およびデータロガー}\par
恒温槽内で加熱している試料の内部温度および槽内温度を測定するために,K型熱電対とGRAPHTEC製データロガーGL240を使用した.温度は20秒間隔で記録し,フィルタ値は5に設定した.本実験ではK型熱電対を4本使用した.2台の恒温槽を用いたため,Ch1を1台目の恒温槽内の温度測定に,Ch2を1台目の恒温槽内に静置した試料の内部温度測定に用いた.同様に,Ch3を2台目の恒温槽内の温度測定に,Ch4を2台目の恒温槽内に静置した試料の内部温度測定に使用した.なお,試料内部温度を測定する熱電対は,試料の中心からわずかにずれた位置に挿入した.K型熱電対を試料に挿入した様子を図\ref{fig:熱電対挿入位置}に示す.\par

\begin{figure}[htbp]
  \centering
  \includegraphics[width=60mm]{images/熱電対挿入方法.pdf} %pdfのみ
  \caption{A K-type thermocouple was inserted into a metal container filled with grease. The thermocouple was inserted near the center of the grease but slightly shifted from the central axis of the container. The photograph was taken inside a thermostatic chamber.}\label{fig:熱電対挿入位置} 
\end{figure}\par

\subsection{1/2円錐を用いる不混和ちょう度試験の手順}
本試験で行った具体的なちょう度試験の手順を示す.

\vspace{5mm}
\noindent\textbf{試料の準備}\par
本試験では,同一試料に対して4つの金属容器を用意した.\par
試料を容器に半分程度充填した後,金属容器を机に叩きつけるように落とし,試料中に混入した気泡を除去した.その後,容器に再度試料を追加して満杯とし,再充填と気泡除去操作を複数回繰り返すことで,試料を密実に充填した.この際,試料をかき混ぜたり攪拌したりしないように注意を払った.\par
試料表面の整形には,へらを試料面に対して約$45\,\tcdegree$の角度で傾け,金属容器の上縁に沿って滑らせながら余分な試料を除去し,表面を平滑に整えた.整形後は,測定までの間,試料表面にへら等で触れないように取り扱った.\par
以上の手順で準備した4つの試料のうち,1つにはK型熱電対を挿入し,内部温度が目標値に達しているかを確認するために使用した.残りの3つの容器をちょう度測定に用いた.\par

\vspace{5mm}
\noindent\textbf{測定前の恒温処理手順}\par
準備した4つの金属容器を恒温槽に静置し,運転を開始した.\par
24時間後,熱電対により試料内部の温度が目標値に達したことを確認し,ちょう度測定を実施した.本試験では,JIS K 2220に規定されている25\,${}^\circ$Cに加え,40\,${}^\circ$Cから160\,${}^\circ$Cまで,20\,${}^\circ$C刻みで温度条件を設定し,測定を行った.\par

\vspace{5mm}
\noindent\textbf{ちょう度測定手順}\par
恒温槽内の試料が目標温度に達していることをデータロガーで確認後,試料を素早く取り出し,図\ref{fig:ちょう度測定}に示すV字型の位置固定具に金属容器を設置した.\par
ちょう度が97を超える試料では,1/2円錐を金属容器の中心に1回だけ落下させ,1容器あたり1回のみ測定を行った\cite{ちょう度JIS}.一方で,ちょう度が97以下の試料については,1つの容器内で3回の測定を実施した\cite{ちょう度JIS}.この場合,測定位置は金属容器の中心から外側に向かって等間隔に3点を選び,互いに約$120\,\tcdegree$間隔となるよう配置した.また,円錐が容器の縁や前回の測定跡に干渉しないよう十分に注意して測定を行った\cite{ちょう度JIS}.測定位置の例を図\ref{fig:ちょう度測定位置}に示す.\par
位置決め完了後,図\ref{fig:ちょう度測定}に示す円錐につながる保持軸を手で支えて,円錐の先端が試料表面に触れる直前の高さに静止させた.その後,手を離して円錐を自由落下させ,落下開始から$5\pm0.5$秒後に保持軸を持ち上げて円錐を回収し,円錐の真横からデジタルカメラで撮影を行った.\par
この手順を,残り2つの試料を充填した金属容器についても同様に実施した.\par

\begin{figure}[htp]
  \centering
  \includegraphics[width=120mm]{images/ちょう度測定位置.pdf} %pdfのみ
  \caption{Measurement position for samples with a penetration of 97 or less.}\label{fig:ちょう度測定位置} 
\end{figure}\par

\vspace{5mm}
\noindent\textbf{洗浄および再使用}\par
試験終了後,金属容器から試料を取り出し,ハンマーを用いて容器を円筒部と底板に分離した.その後,円錐・円筒部・底板・金属へらをアセトンに一定時間浸漬し,油分がある程度除去された時点で,ブラシを用いて洗浄した.洗浄後は各部品を乾燥させ,円筒部と底板を瞬間接着剤で再接着し,次の実験に金属容器として再使用した.接着の際には,円筒と底板の間に隙間が生じないよう十分に注意した.\par

\vspace{5mm}
\noindent\textbf{円錐の進入深さの計測}\par
実験終了後,デジタルカメラで撮影した画像をImageJに取り込み,1画像あたり3回,1条件につき計9回の進入深さを測定した.計測精度は,0.1\,mm単位とした\cite{ちょう度JIS}.\par

\vspace{5mm}
\noindent\textbf{不混和ちょう度の計算方法}\par
円錐の進入深さのデータを10倍して得られた値の平均を,ちょう度とした\cite{ちょう度JIS}.得られたちょう度は,JIS Z 8401規則B\cite{丸め方}に従い,丸め幅1で丸めた.


\section{グリースの密着強度測定試験}
本試験では,転がり軸受用グリースとナイロン66との密着強度を評価するため,グリース内に埋め込まれたナイロン66製円筒を引き上げる引張試験を実施した.

\subsection{グリースの密着強度試験の原理}
本試験では,試料で満たした金属容器の中心にナイロン66製円筒を垂直に静置し,その後,電動スライダを用いて円筒を一定速度で鉛直方向に引き上げた.円筒が静止状態から移動を開始した時点以降の引き上げ力をロードセルにより計測し,この荷重をグリースと円筒表面との密着強度評価に用いた.\par
さらに,円筒を容器外まで引き上げた後も,グリースの粘性により円筒下端と容器内のグリースが糸を引くように連結する現象が観察された.この連結が完全に切断される瞬間の荷重も計測対象とした.なお,米国の大学における先行研究では,この切断時の荷重を密着強度として定義しており,本研究でも同様の観点から評価を行った.\par
得られた密着強度は,それぞれの荷重を円筒表面のグリース付着面積で除した応力として算出した.


\subsection{密着強度測定の試験装置および器具}
図\ref{fig:密着強度測定}にグリースとナイロン66の密着強度試験装置の外観を示す.装置は,ナイロン66製円筒の引上げ機構と荷重計測器で構成される.円筒の引上げは,プーリを介して電動スライダを用い,定速で引上げを行った.荷重の測定は,微荷重ロードセルを電動スライダに取付けて使用した.

\begin{figure}[htbp]
  \centering
  \includegraphics[width=130mm]{images/密着強度測定.pdf} %pdfのみ
  \caption{External view of the Adhesion strength testing device.}\label{fig:密着強度測定} 
\end{figure}\par

\vspace{5mm}
\noindent\textbf{ナイロン66製円筒}\par
グリースとナイロン66との密着強度を測定するための円筒は,図\ref{fig:ナイロン円筒}に示す形状とした.引上げに使用する糸を付けるために,円筒上部には,アイボルトを取付けた.円筒とアイボルトを含めた全体の質量は,20.30\,gで,外観図を図\ref{fig:円筒外観図}に示す.

\begin{figure}[htbp]
  \centering
  \includegraphics[width=155mm]{images/ナイロン円筒.pdf} %pdfのみ
  \caption{Shape of the nylon cylinder.}\label{fig:ナイロン円筒} 
\end{figure}\par

\begin{figure}[htbp]
  \centering
  \includegraphics[width=80mm]{images/円筒外観図.pdf} %pdfのみ
  \caption{Shape of the nylon cylinder. The blue part is made of nylon 66, and a stainless steel eye bolt is fastened with a screw.}\label{fig:円筒外観図} 
\end{figure}\par

\newpage
\vspace{5mm}
\noindent\textbf{試料を入れる金属容器}\par
試料(グリース)を充填する金属容器は,図\ref{fig:金属容器}に示す円筒と,一部を加工した底板を瞬間接着剤で隙間なく接合して作製した.底板中央には,ナイロン66製円筒を垂直に設置できるよう,図\ref{fig:密着強度_底板}に示すような深さ1\,mmの溝を加工した.\par

\begin{figure}[htbp]
  \centering
  \includegraphics[width=90mm]{images/密着強度_底板.pdf} %pdfのみ
  \caption{Bottom plate of the metal container used for adhesion strength test. The cylindrical part is the same as that shown in Fig.\ref{fig:金属容器}, and a 1 mm groove was machined at the center to vertically position the nylon 66 cylinder.}\label{fig:密着強度_底板} 
\end{figure}\par

\vspace{5mm}
\noindent\textbf{円筒引上げ機構}\par
ナイロン製円筒を糸で引き上げる機構である.引上げにはオリエンタルモータ製電動スライダEZSM3E025AZMCを用いた.
図\ref{fig:円筒外観図}に示すように,プーリを介して垂直方向の引上げ動作を水平方向の運動に変換した.

\vspace{5mm}
\noindent\textbf{ロードセル}\par
本実験に用いるロードセルの選定にあたり,イマダ製プッシュプルスケールPS-500Nを用いて,グリースとナイロンの密着強度を簡易的に評価する引張試験を行った.図\ref{fig:密着強度予備試験}に示すように,図\ref{fig:円筒外観図}の円筒とプッシュプルスケールを黄色い糸で結び,予備引上げ試験を実施した.

\begin{figure}[htbp]
  \centering
  includegraphics[width=60mm]{images/密着強度予備試験.pdf}
  \caption{Preliminary adhesion strength test.}\label{fig:密着強度予備試験} 
\end{figure}\par

その結果,最大引張荷重は約3\,Nであった.この結果に基づき,図\ref{fig:密着強度測定}に示す密着強度測定器には,10\,Nまで測定可能なオリエンタルモータ製微荷重ロードセルLVS-1KAを採用した.\par


\subsection{グリースの密着強度試験の手順}






つまり、図\ref{tab:sample}により,

\begin{align}
\label{eq:formula_2}
    {
    N = \frac{1}{2} \rho {V_a}^2 {C_N\alpha}\cdot \alpha S
    }
\end{align}
